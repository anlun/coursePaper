\newpage

\subsection{Библиотека Свиерстры}

рассказать, что является оптимальной раскладкой, с точки зрения библиотеки

Библиотека Азеро и Свиерстры, описанная в \cite{swierstra}, отличается от предыдущих библиотек тем, что дает возможность явным образом задать несколько несвязанных вариантов раскладки документа, так как там есть оператор \textbf{(<|>)}. Этот оператор обладает следующей сигнатурой:
\inputminted{haskell}{codes/chooseSw.hs}
Этот оператор берет два документа и создает документ, который при раскладке может стать первым или вторым документом, в зависимости от того, какой из документов раскладывается оптимальней. Наличие такого оператора сразу же решает проблему со скобкой, которая была поднята в обзоре библиотеки Хьюза.